\documentclass[11pt,]{article}
\usepackage[]{mathpazo}
\usepackage{amssymb,amsmath}
\usepackage{ifxetex,ifluatex}
\usepackage{fixltx2e} % provides \textsubscript
\ifnum 0\ifxetex 1\fi\ifluatex 1\fi=0 % if pdftex
  \usepackage[T1]{fontenc}
  \usepackage[utf8]{inputenc}
\else % if luatex or xelatex
  \ifxetex
    \usepackage{mathspec}
  \else
    \usepackage{fontspec}
  \fi
  \defaultfontfeatures{Ligatures=TeX,Scale=MatchLowercase}
\fi
% use upquote if available, for straight quotes in verbatim environments
\IfFileExists{upquote.sty}{\usepackage{upquote}}{}
% use microtype if available
\IfFileExists{microtype.sty}{%
\usepackage{microtype}
\UseMicrotypeSet[protrusion]{basicmath} % disable protrusion for tt fonts
}{}
\usepackage[margin = 1in]{geometry}
\usepackage{hyperref}
\hypersetup{unicode=true,
            pdftitle={Pacing Profiles in 2000 Meter World Championship Rowing},
            pdfborder={0 0 0},
            breaklinks=true}
\urlstyle{same}  % don't use monospace font for urls
\usepackage{longtable,booktabs}
\usepackage{graphicx,grffile}
\makeatletter
\def\maxwidth{\ifdim\Gin@nat@width>\linewidth\linewidth\else\Gin@nat@width\fi}
\def\maxheight{\ifdim\Gin@nat@height>\textheight\textheight\else\Gin@nat@height\fi}
\makeatother
% Scale images if necessary, so that they will not overflow the page
% margins by default, and it is still possible to overwrite the defaults
% using explicit options in \includegraphics[width, height, ...]{}
\setkeys{Gin}{width=\maxwidth,height=\maxheight,keepaspectratio}
\IfFileExists{parskip.sty}{%
\usepackage{parskip}
}{% else
\setlength{\parindent}{0pt}
\setlength{\parskip}{6pt plus 2pt minus 1pt}
}
\setlength{\emergencystretch}{3em}  % prevent overfull lines
\providecommand{\tightlist}{%
  \setlength{\itemsep}{0pt}\setlength{\parskip}{0pt}}
\setcounter{secnumdepth}{5}
% Redefines (sub)paragraphs to behave more like sections
\ifx\paragraph\undefined\else
\let\oldparagraph\paragraph
\renewcommand{\paragraph}[1]{\oldparagraph{#1}\mbox{}}
\fi
\ifx\subparagraph\undefined\else
\let\oldsubparagraph\subparagraph
\renewcommand{\subparagraph}[1]{\oldsubparagraph{#1}\mbox{}}
\fi

%%% Use protect on footnotes to avoid problems with footnotes in titles
\let\rmarkdownfootnote\footnote%
\def\footnote{\protect\rmarkdownfootnote}

%%% Change title format to be more compact
\usepackage{titling}

% Create subtitle command for use in maketitle
\newcommand{\subtitle}[1]{
  \posttitle{
    \begin{center}\large#1\end{center}
    }
}

\setlength{\droptitle}{-2em}

  \title{Pacing Profiles in 2000 Meter World Championship Rowing}
    \pretitle{\vspace{\droptitle}\centering\huge}
  \posttitle{\par}
    \author{}
    \preauthor{}\postauthor{}
      \predate{\centering\large\emph}
  \postdate{\par}
    \date{2018-09-16}


\begin{document}
\maketitle
\begin{abstract}
The pacing strategy adopted by an athlete(s) is one of the major
determinants of successful performance during timed competition. Various
pacing profiles are reported in the literature and its potential to
precede a winning performance depends on the mode of sport. However, in
2000m rowing, the definition of these pacing profiles has been
subjective and there is a need to be more objective with the definition.
Our aim is to objectively identify pacing profiles used in World
Championship 2000m rowing races. To do this the average speed and stroke
rate (SR) for each 50m split for each boat in every race of the Rowing
World Championships from 2010-2017 was Scraped from www.worldrowing.com.
Pacing profiles are determined by using k-shape clustering on the
average boat speeds at each 50m split. Finally, clusters are described
using boat and race descriptors to draw conclusions about who, when and
why each pacing profile was observed. Chi-Squared Tests of Independence
with Bonferroni corrections are used to test whether variables such as
boat size, gender, round, or rank are associated with pacing profiles.
Four pacing strategies (Allout, Even, J-Shaped, and Reverse J-Shaped)
are identified from the clustering process. Boat size, round (Heat vs
Finals) and rank are all found to affect pacing profiles. Whereas,
gender, and weight class do not affect a boat's pacing profile. This
novel approach of using clustering is able to objectively define four
strategies used in 2000m rowing competitions. \par \textbf{Keywords:}
rowing, pacing profiles, k-shape clustering, race analysis
\end{abstract}

\pagebreak

\section{Introduction}\label{introduction}

Across ``closed-loop'' design sports, competitions where athlete(s)
attempt to complete a set distance in the shortest time (Abbiss \&
Laursen, 2008), there have been differnt pacing strategies that have
been identified. Most of these pacing strategies have been defined in
running and cycling races and an attempt has been made to define these
strategies in 2000m rowing (Muehlbauer, Schindler, \& Widmer, 2010)
(Michael D. Kennedy \& Bell, 2003) (Muehlbauer \& Melges, 2011)
(Garland, 2005). However, these attempts approach the problems in
different ways and come to different conlclusions. We attempt to
standardize the definition of pacing profiles in rowing by using more
granual data than other attempts.

\subsection{Pacing Profiles in Rowing}\label{pacing-profiles-in-rowing}

Determining optimal pacing profiles can be done using ergometric data
(Michael D. Kennedy \& Bell, 2003) or by using observational data from
actual competitions (Garland, 2005) (Muehlbauer et al., 2010)
(Muehlbauer \& Melges, 2011).

In 2003 Kennedy and Bell used simulated rowing and training results to
suggest that there was a different optimal race profiles for different
genders. They found that a constant pacing profile was optimal for
males, and an all-out profile was optimal for females.

In 2004, Garland used observational data from 2000 Olympic, 2001 World
Championship and 2001 \& 2002 British indoor Rowing Championship
competitions. His analysis found that when using 4 time splits measured
every 500 meters that men and women show no difference in their observed
pacing strategies. Garland eliminated races that showed signs of
slowdowns from the analysis. Then Meuhlbauer et. al. in 2010 and
Meuhlbauer and Melges in 2011 used the same type of split time data to
model pacing profiles. In 2010 they found that gender, round of race
(whether race was in qualifying heat or the final race for the
category), size of boat, coxed, and sculle did not affect pacing
strateiges for the 2008 Olympics. In 2011, they had a different finding
that indicated that round of race did effect pacing profiles in World
Championship races between 2001 and 2009. They performed these analyses
by fitting linear quadratic models to the four time splits.

\subsection{Pacing Profiles in other Closed-Loop
Sports}\label{pacing-profiles-in-other-closed-loop-sports}

In other races of fixed distance like cycling, and running six pacing
profiles have been defined (Abbiss \& Laursen, 2008). The six profiles
are Negative, All-Out, Positive, Even, Parabolic-Shaped, and Variable
pacing.

A negative-split pacing profile, is defined by an increase in speed
across splits (which result in smaller relative split times as the race
progresses) and is often used in middle-distance events.

An All-Out profile is used when the it is believed that energy reserves
are best distributed at the start of the race. This is commonly found in
shorter events like the 100 meter sprint. This will often result in
``negative-split'' times in shorter events, and ``positive-split'' times
in longer events.

A positive pacing profile is one where the athlete's speed decreases
through each split in the event. This is found often in swimming, where
the dive start allows athletes to reach their maximal speed quickly.

Even pacing profiles are categorized by a relatively small portion of
the race spend in the acceleration phase and the majority of the race at
a constant pace.

There are 2 sub pacing strategies for Parabolic-Shaped pacing profiles.
J-Shaped, Reverse J-shaped and U shaped. In general these strategies
follow a parabolic shape where the middle fo the race sees the lowest
relative speeds. In the U shape strategy, the start and end of the race
see the same relative speed. The J-Shaped strategy has a greater
relative speed at the end of the race while the Reverse J-Shaped
profiles has a greater relative speed at the start of the race.

The last profile mentioned by Abbiss and Laursen is Variable Pacing. It
is a strategy that is used to adapt to changing conditions in the race
course, like uphills and downhills in cycling.

\subsection{Our Approach}\label{our-approach}

The classification of pacing profiles has usually been approached by
fitting linear models to 4 split times. We believe that using more
granular data describing a boat's speed throughout the race will be able
to paint a better picture of how the boat is performing throughout the
race. We also believe that using a clustering technique to classify
similarly shaped speed curves together will provide a novel approach to
defining pacing profiles. We then hope to match the clusters we find to
pacing profiles mentioned in racing literature. Furthermore, we then
plan on determining which race factors affect the use of a pacing
profile and which do not.

To do so we present the following:

\begin{enumerate}
\def\labelenumi{\arabic{enumi}.}
\item
  A github repository with Global Positioning Sytem (GPS), Media Start
  List, and Race Results data from World Championships from 2010 to
  2017. Additionally, we have included the code needed to scrape this
  data for future years and replicate our process of scraping and
  extracting the current data.
\item
  A novel approach of classifying pacing profiles for boats in 200m
  rowing.
\item
  Which race factors affect and do not affect the use of a given pacing
  profile during an event.
\end{enumerate}

\section{Data Collection}\label{data-collection}

The availability of data in rowing is limited. While some studies have
data from ergometric machines (Michael D. Kennedy \& Bell, 2003) and
others have split time results (Garland, 2005) (Muehlbauer et al., 2010)
(Muehlbauer \& Melges, 2011), there was been limited use of the publicly
available GPS race data. This is because the data is not stored in a
local easy to use area.

GPS data for almost every World Championship race from 2010 to 2017 is
available in Portable Document Fomat (PDF) files that are located in the
summary of each round of each category of each World Championship. Not
only was it previously a long process to download each pdf file, but it
was a long manual process to copy and paste the data from each file into
a more convenient form. Our data collection process involved writing a
bash script to scrape all GPS PDFs, along with Media Start List PDFs and
Race Results PDFs from every round of every category of the specified
World Championships. This process takes roguhly 25 minutes to run. In
total there were 5322 PDFs that were scraped from the website. 1 of
these files was a broken link so it left us with 5321 PDF files from the
8 years of World Championships.

\begin{longtable}[]{@{}cc@{}}
\caption{Number of PDF Files by World Championship Year. The number of
races fluctuates by year depending on whether it was an Olympic year, if
Paralympic rowing was included and whether Junior World Championships
were included.}\tabularnewline
\toprule
\begin{minipage}[b]{0.09\columnwidth}\centering\strut
Year\strut
\end{minipage} & \begin{minipage}[b]{0.27\columnwidth}\centering\strut
Number of PDF Files\strut
\end{minipage}\tabularnewline
\midrule
\endfirsthead
\toprule
\begin{minipage}[b]{0.09\columnwidth}\centering\strut
Year\strut
\end{minipage} & \begin{minipage}[b]{0.27\columnwidth}\centering\strut
Number of PDF Files\strut
\end{minipage}\tabularnewline
\midrule
\endhead
\begin{minipage}[t]{0.09\columnwidth}\centering\strut
2010\strut
\end{minipage} & \begin{minipage}[t]{0.27\columnwidth}\centering\strut
0\strut
\end{minipage}\tabularnewline
\begin{minipage}[t]{0.09\columnwidth}\centering\strut
2011\strut
\end{minipage} & \begin{minipage}[t]{0.27\columnwidth}\centering\strut
0\strut
\end{minipage}\tabularnewline
\begin{minipage}[t]{0.09\columnwidth}\centering\strut
2012\strut
\end{minipage} & \begin{minipage}[t]{0.27\columnwidth}\centering\strut
0\strut
\end{minipage}\tabularnewline
\begin{minipage}[t]{0.09\columnwidth}\centering\strut
2013\strut
\end{minipage} & \begin{minipage}[t]{0.27\columnwidth}\centering\strut
0\strut
\end{minipage}\tabularnewline
\begin{minipage}[t]{0.09\columnwidth}\centering\strut
2014\strut
\end{minipage} & \begin{minipage}[t]{0.27\columnwidth}\centering\strut
0\strut
\end{minipage}\tabularnewline
\begin{minipage}[t]{0.09\columnwidth}\centering\strut
2015\strut
\end{minipage} & \begin{minipage}[t]{0.27\columnwidth}\centering\strut
0\strut
\end{minipage}\tabularnewline
\begin{minipage}[t]{0.09\columnwidth}\centering\strut
2016\strut
\end{minipage} & \begin{minipage}[t]{0.27\columnwidth}\centering\strut
0\strut
\end{minipage}\tabularnewline
\begin{minipage}[t]{0.09\columnwidth}\centering\strut
2017\strut
\end{minipage} & \begin{minipage}[t]{0.27\columnwidth}\centering\strut
0\strut
\end{minipage}\tabularnewline
\bottomrule
\end{longtable}

The next step to create our dataset was to extract the information from
the PDF files. Some races were missing, one or more of the GPS, Results,
or Media Start List data. We did not parse any PDFs for races that were
missing a file. This left us with 1,736 unique races over the 8 years of
World Championships. We developped 3 separate PDF extractor functions
and 1 main pdf extractor driver to parse each PDF and merge together
information based on regular expressions and consitent locations of data
within the PDFs. This process had to accomodate changes in naming
conventions, changes to the location of information and addition of
information that occured over the years.

From the GPS PDF we extract average speeds and strokes per minute for
each 50 meter split (sometimes smaller splits in later years) for each
boat along with the race date, race nuber, round type, country
abbreviation and lane number.

From the Results PDF we extract the finishing rank, the progression
(next race or finishing tournament rank), 500 meter split times, whether
the boat ``Did not Start'', ``Did not Finish'' or was ``Excluded'' from
the race, the country abbreivation and lane number.

From the Media Start List PDF we extract the names, birthdays and
positions of each boat member as well as the country abbreviation and
lane number

We use parallel computing to speed up the process of extracting the
information for the 5321 pdfs. Using 4 cores instead of one we see the
time to extract files drop from 58 minutes to 24 minutes.

Finally we augment the data by breaking down the race's cateogry
abbreviation. We can split each race's category into Weight Class (Light
or Open), Gender (Male, Female or Mixed), Size (1, 2, 4, 8), Discipline
(Scull or Sweep), Adaptive (True, or False), Adaptive Designation (Arm
and Shoulders, Intelectual Disabitliy, Trunk and Arm, Leg Trunk and Arm,
or None), Age Group (Junior or Senior), Race Round (Exhibition,
Repecharge, Heat, Quarterfinal, Semifinal, or Final) and a Qualifier or
Final designation.

This left us with a dataset of 9264 boats' races (rows) and 131
variables.

\section{Data Filtering and Pre
Processing}\label{data-filtering-and-pre-processing}

Before beginning any clustering procedures we needed to clean the data
set and keep only the boats' racesthat we wanted to cluster. Some races
have GPS data errors where the reported average speed is lower than the
true average speed. In other cases the average speed is simply not
reported. We remove any boat that has a unreported average speed at any
of the split measurements (at every 50 meters) and any boat that saw
reported average speed less than 2 meters per second. Additionally, we
removed any boats that received ``Did not Starts'', ``Did not Finishes''
or ``Exclusions''. This reduced the number of boats' races from 9264 to
8170.

To determine pacing profiles raw speeds at each split are often compared
to the mean speed of a boat throughout the race (Garland, 2005). So we
define \(x_{i,j}\), as the speed at split \(i\) for boat \(j\) and
normalize to get \(y_{i,j}\).

\begin{align*}
    y_{i,j} = \frac{x_{i,j} - \bar{x}_{j}}{\sigma_{j}}
\end{align*}

This is useful because the magnitude of the speed has been normalized
and we can now compare the pacing profile of an 8 person boat to that of
a 1 person boat while accounting for the fact that their speeds will
have different magnitudes.

\section{Clustering Pacing Profiles}\label{clustering-pacing-profiles}

With the boats' races with errors removed and speeds standardize to the
same scale we can begin the clustering process. The idea is that we
would like to group velocity curves of similar shape together. There is
a large literature in clustering and the area of longitudinal clustering
is growing. McNicholas and Subedi used a model based clustering approach
that uses mixtures of multivariate t-distributions with a linear model
for the mean and a modified Cholesky-decomposed covariance structure to
cluster gene expressions (McNicholas, Sanjeena, \& Subedi, 2012).
Additionally, Kumar and Futschik used a soft clustering technique to
cluster the shapes of microarray data (Kumar \& Futschik, 2007).
Finally, using UCR time-series datasets (Chen et al., 2015), a
collection of datasets that has been collected to test clustering
techniques and improve the clustering techniques that are published,
Paparrizos and Gravano developped the k-Shape clustering technique for
time series data (Paparrizos \& Gravano, 2016).

\subsection{K-Shape Clustering}\label{k-shape-clustering}

In K-Shape clustering a new distance method, called ``Shape-based
distance (SBD)'' and a new method for computing centroids. When SBD is
evaluated against other distance metrics such as Dynamic Time Warping,
it reaches similar error rates on the UCR datasets but much more
efficient run times.

The K-Shape algorithm is implemented in the dtwclust package
(Sarda-Espinosa, 2018). In it's implementation it normalizes the columns
to the same scale. So it takes our \(y_{i,j}\) defined above and
transforms it to \(z_{i,j}\) defined below.

\begin{align*}
    z_{i,j} = \frac{y_{i,j} - \bar{y}_{i}}{\sigma_{i}}
\end{align*}

K-Shape then functions very similarly to the k-means algorithm (Lloyd,
1982) in the way that it uses an interative refinement that minimizes a
given distance function.

\section{Results}\label{results}

Our first round of results

\subsection{Identifying Pacing
Profiles}\label{identifying-pacing-profiles}

\subsection{Relationship between Boats and Pacing
Profiles}\label{relationship-between-boats-and-pacing-profiles}

\subsubsection{Size of Boats}\label{size-of-boats}

\subsubsection{Round of Race}\label{round-of-race}

\subsubsection{Final Placement}\label{final-placement}

\subsubsection{1st Half Rankings}\label{st-half-rankings}

\subsubsection{Gender}\label{gender}

\subsubsection{Weight Class}\label{weight-class}

\subsubsection{Stroke Rate}\label{stroke-rate}

\section{Conclusions}\label{conclusions}

\section{Future Research}\label{future-research}

\subsection{Data Accessibility}\label{data-accessibility}

\subsection{Extending the Methods}\label{extending-the-methods}

\section{Acknowledgements}\label{acknowledgements}

\pagebreak

\section*{References}\label{references}
\addcontentsline{toc}{section}{References}

\hypertarget{refs}{}
\hypertarget{ref-Abbiss}{}
Abbiss, C. R., \& Laursen, P. B. (2008). Describing and understanding
pacing strategies during athletic competition, \emph{38}, 239--52.

\hypertarget{ref-UCR}{}
Chen, Y., Keogh, E., Hu, B., Begum, N., Bagnall, A., Mueen, A., \&
Batista, G. (2015). \emph{The ucr time series classification archive}.
Retrieved from \url{www.cs.ucr.edu/~eamonn/time_series_data/}

\hypertarget{ref-remotes}{}
Csárdi, G., Wickham, H., Chang, W., Hester, J., Morgan, M., \&
Tenenbaum, D. (2017). \emph{Remotes: R package installation from remote
repositories, including 'github'}. Retrieved from
\url{https://CRAN.R-project.org/package=remotes}

\hypertarget{ref-Garland}{}
Garland, S. W. (2005). An analysis of the pacing strategy adopted by
elite competitors in 2000 m rowing. \emph{British Journal of Sports
Medicine}, \emph{39}(1), 39--42. British Association of Sport; Excercise
Medicine. Retrieved from \url{https://bjsm.bmj.com/content/39/1/39}

\hypertarget{ref-lubridate}{}
Grolemund, G., \& Wickham, H. (2011). Dates and times made easy with
lubridate. \emph{Journal of Statistical Software}, \emph{40}(3), 1--25.
Retrieved from \url{http://www.jstatsoft.org/v40/i03/}

\hypertarget{ref-glue}{}
Hester, J. (2018). \emph{Glue: Interpreted string literals}. Retrieved
from \url{https://CRAN.R-project.org/package=glue}

\hypertarget{ref-Kumar}{}
Kumar, L., \& Futschik, M. (2007). Kumar l, futschik e.. mfuzz: A
software package for soft clustering of microarray data. bioinformation
2: 5-7, \emph{2}, 5--7.

\hypertarget{ref-tabulizer}{}
Leeper, T. J. (2018). \emph{Tabulizer: Bindings for tabula pdf table
extractor library}.

\hypertarget{ref-kmeans}{}
Lloyd, S. P. (1982). Least squares quantization in pcm. \emph{IEEE
Transactions on Information Theory}, \emph{28}, 129--137.

\hypertarget{ref-McNicholas}{}
McNicholas, P., Sanjeena, \& Subedi. (2012). Clustering gene expression
time course data using mixtures of multivariate t-distributions,
\emph{142}, 1114--1127.

\hypertarget{ref-Kennedy}{}
Michael D. Kennedy, \& Bell, G. J. (2003). Development of race profiles
for the performance of a simulated 2000-m rowing race. \emph{Can J Appl
Physiol}, \emph{28}(4), 536--546.

\hypertarget{ref-Muehlbauer_two}{}
Muehlbauer, T., \& Melges, T. (2011). Pacing patterns in competitive
rowing adopted in different race categories. \emph{The Journal of
Strength \& Conditioning Research}, \emph{25}. Retrieved from
\url{https://journals.lww.com/nsca-jscr/Fulltext/2011/05000/Pacing_Patterns_in_Competitive_Rowing_Adopted_in.15.aspx}

\hypertarget{ref-Muehlbauer_one}{}
Muehlbauer, T., Schindler, C., \& Widmer, A. (2010). Pacing pattern and
performance during the 2008 olympic rowing regatta. \emph{European
Journal of Sport Science}, \emph{10}(5), 291--296. Routledge. Retrieved
from \url{https://doi.org/10.1080/17461390903426659}

\hypertarget{ref-pdftools}{}
Ooms, J. (2018). \emph{Pdftools: Text extraction, rendering and
converting of pdf documents}. Retrieved from
\url{https://CRAN.R-project.org/package=pdftools}

\hypertarget{ref-Paparrizos}{}
Paparrizos, J., \& Gravano, L. (2016). K-shape: Efficient and accurate
clustering of time series. \emph{SIGMOD Rec.}, \emph{45}(1), 69--76. New
York, NY, USA: ACM. Retrieved from
\url{http://doi.acm.org/10.1145/2949741.2949758}

\hypertarget{ref-R}{}
R Core Team. (2018). \emph{R: A language and environment for statistical
computing}. Vienna, Austria: R Foundation for Statistical Computing.
Retrieved from \url{https://www.R-project.org/}

\hypertarget{ref-dtwclust}{}
Sarda-Espinosa, A. (2018). \emph{Dtwclust: Time series clustering along
with optimizations for the dynamic time warping distance}. Retrieved
from \url{https://CRAN.R-project.org/package=dtwclust}

\hypertarget{ref-rJava}{}
Urbanek, S. (2018). \emph{RJava: Low-level r to java interface}.
Retrieved from \url{https://CRAN.R-project.org/package=rJava}

\hypertarget{ref-ggplot2}{}
Wickham, H. (2016). \emph{Ggplot2: Elegant graphics for data analysis}.
Springer-Verlag New York. Retrieved from \url{http://ggplot2.org}

\hypertarget{ref-tidyverse}{}
Wickham, H. (2017a). \emph{Tidyverse: Easily install and load the
'tidyverse'}. Retrieved from
\url{https://CRAN.R-project.org/package=tidyverse}

\hypertarget{ref-multidplyr}{}
Wickham, H. (2017b). \emph{Multidplyr: Partitioned data frames for
'dplyr'}. Retrieved from \url{https://github.com/hadley/multidplyr}

\hypertarget{ref-devtools}{}
Wickham, H., Hester, J., \& Chang, W. (2018). \emph{Devtools: Tools to
make developing r packages easier}. Retrieved from
\url{https://CRAN.R-project.org/package=devtools}

\hypertarget{ref-knitr}{}
Xie, Y. (2014). Knitr: A comprehensive tool for reproducible research in
R. In V. Stodden, F. Leisch, \& R. D. Peng (Eds.), \emph{Implementing
reproducible computational research}. Chapman; Hall/CRC. Retrieved from
\url{http://www.crcpress.com/product/isbn/9781466561595}


\end{document}
